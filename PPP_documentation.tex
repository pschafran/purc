%% Based on the GSI document, so mostly from Swofford's pre-existing styles

%% Remember to compile as XeLaTeX (rather than just LaTeX)

% Simple command-line command to get a word count breakdown: texcount mytexfile.tex

% quick command refs:
%\textrm   \rmfamily   Roman family
%\textsf   \sffamily   Sans serif family
%\texttt   \ttfamily   Typewriter family
%\textup   \upshape    Upright shape
%\textit   \itshape    Italic shape
%\textsl   \slshape    Slanted shape
%\textsc   \scshape    Small caps shape
%\textmd   \mdseries   Medium series
%\textbf   \bfseries   Boldface series

\documentclass[12pt,letterpaper]{article}

\usepackage{fixltx2e}
\usepackage{textcomp}
\usepackage{fullpage}
\usepackage{amsfonts}
\usepackage{verbatim}
\usepackage[english]{babel}
\usepackage{pifont}
\usepackage{color}
\usepackage{setspace}
\usepackage{textcase}
\usepackage{lscape}
\usepackage{indentfirst}
\usepackage[normalem]{ulem}
\usepackage{booktabs}
%\usepackage{nag}
\usepackage{xspace}

\usepackage{natbib}
\bibliographystyle{evolution}

\usepackage{float}
\usepackage{latexsym}
%\usepackage{hyperref}
\usepackage{url}
%\usepackage{html}
\usepackage{hyperref}
\usepackage{epsfig}
\usepackage{graphicx}
\usepackage{amssymb}
\usepackage{amsmath}
\usepackage{bm}
\usepackage{array}
\usepackage[version=3]{mhchem} % Without the version= this threw up an error when I typeset
\usepackage{ifthen}
\usepackage{caption}
\usepackage{hyperref}
%\usepackage{xcolor}
\usepackage{amsthm}
\usepackage{amstext}

\usepackage{newicktree}

%DLS: My standard font stuff...
\ifxetex
	\usepackage{xltxtra,xunicode}
	\defaultfontfeatures{Mapping=tex-text}
	\setromanfont[Mapping=tex-text]{Palatino}
	\setsansfont[Scale=MatchLowercase,Mapping=tex-text]{Gill Sans}
	\setmonofont[Scale=MatchLowercase]{Andale Mono}
	\usepackage{unicode-math}
%	\setmathfont{xits-math.otf}
	\setmathfont{Asana-Math.otf}
	\newfontfamily\applekeyfont{Lucida Grande}
	\newcommand\cmdkey[1]{{\applekeyfont \char"2318 #1}}
	\newcommand\optionkey[1]{{\applekeyfont \char"2325 #1}}
	\newcommand\shiftkey[1]{{\applekeyfont \char"21E7 #1}}
	\newcommand\tabkey[0]{{\applekeyfont \char"21E5}}
	\newcommand\deletekey[0]{{\applekeyfont \char"232B}}
\else
	\usepackage[utf8]{inputenc}
	\usepackage[T1]{fontenc}
\fi

%-------------- Begin formatting for 'Evolution' --------------
\setsansfont[Scale=MatchLowercase,Mapping=tex-text]{Helvetica}
\newcommand{\figref}[1]{Fig. \ref{#1}}

\renewcommand{\abstract}[1]{
\small
\textbf{
\textsf{#1}}
}

\renewcommand{\section}[1]{
\bigskip
\noindent
\begin{Large}
\textit{\textsf{#1}}
\medskip
\end{Large}
}

\renewcommand{\subsection}[1]{
\bigskip
\noindent
\textsf{\textbf{\MakeUppercase{#1}}}
}

%\renewcommand{\subsubsection}[1]{
%\vspace{2ex}
%\noindent
%\textit{#1.}---}

%-------------- End formatting for 'Evolution' --------------


%DLS: paralist is a package for some alternative list environments
\usepackage{paralist}

% DLS: I use the todonotes package to mark up text with 'to do' items.  I add the 'todoin'
%      command for an inline todo with a different appearance.
\usepackage{todonotes}
\newcommand{\todoin}[1]{
  {\singlespacing\todo[inline, size=\small, caption={2do},backgroundcolor=blue!5!white, bordercolor=blue]{#1}}
}

\linespread{1.66}
\raggedright
\setlength{\parindent}{0.5in}

\setcounter{secnumdepth}{0}

\pagestyle{empty}

\renewcommand{\tableofcontents}{}

% ---------------- Begin macros specific to GSI paper ----------------

\newcommand{\gsi}[1]{\ensuremath{GSI(\text{#1})}}

\makeatletter

\newcommand\gs@noargs{\ensuremath{gs}\xspace}
\newcommand\gs@withargs[1][]{\ensuremath{gs(\text{#1})}}
\newcommand\gs[1][]{\ifthenelse{\equal{#1}{}}{\gs@noargs}{\gs@withargs[#1]}}

\newcommand\xgs@noargs[1]{\ensuremath{gs_{#1}}\xspace}
\newcommand\xgs@withargs[2][]{\ensuremath{gs_{#2}(\text{#1})}}
\newcommand\mings[1][]{\ifthenelse{\equal{#1}{}}{\xgs@noargs{min}}{\xgs@withargs[#1]{min}}}
\newcommand\maxgs[1][]{\ifthenelse{\equal{#1}{}}{\xgs@noargs{max}}{\xgs@withargs[#1]{max}}}
\newcommand\obsgs[1][]{\ifthenelse{\equal{#1}{}}{\xgs@noargs{obs}}{\xgs@withargs[#1]{obs}}}
\makeatother

%----------------- End macros specific to GSI paper -----------------


\bibpunct{(}{)}{;}{a}{}{,}  % this is a citation format command for natbib

\begin{document}
\begin{flushright}
Version dated: \today
\end{flushright}
\bigskip
\noindent RH: Sample sizes, null hypotheses, and the GSI

\bigskip
\medskip
\begin{center}

\noindent{\Large \bf \uppercase{What does the genealogical sorting index index, and the perils of
permutation}}
\bigskip


\noindent {\normalsize \bf Carl J. Rothfels$^1$, Matthew G. Johnson$^2$, and David L. Swofford$^3$}\\
\noindent {\small \it
$^1$University Herbarium and Department of Integrative Biology, University of California, Berkeley, California, 94720-2465\\
$^2$Plant Science and Conservation, Chicago Botanic Garden, Glencoe, IL 60022\\
$^3$Department of Biology, Duke University, Durham, NC 27708}\\
\end{center}
\medskip
\noindent{\bf Corresponding author:} Carl J. Rothfels, University of California, Berkeley, California, USA; E-mail: crothfels@yahoo.ca\\

\vspace{0.5in}

{
\setlength{\parindent}{0pt}
\singlespacing
\abstract{Abstract would go in here
} % Having a return after this brace makes a difference

}

\vspace{0.5in}

\noindent \textbf{Keywords:} coalescent, GSI, monophyly, paraphyly, permutation tests,
species delimitation, taxon sampling

\newpage\noindent Example text: Since its introduction \citep{Cummings:2008}, the genealogical
sorting index (GSI) has enjoyed considerable popularity among
evolutionary biologists \citep[e.g.,][]{Welch:2011, Gustafsson:2012,
Weisrock:2010, Niemiller:2011}.

So what is the GSI? It is based on a quantity---``genealogical sorting''
(\gs)---defined as the minimum number of nodes potentially required to
etc

\todoin{TODO: here's how to make a note
\begin{compactitem}
\item first item.
\item second item
\item etc
\end{compactitem}
}

\section{new section}

%%%%%%%%% Acknowledgments
\section{Acknowledgements}

We thank .... NESCent short-term scholar support.. Tree diagrams were
produced with the \textsf{newicktree} \LaTeX package \citep{Savva:2004}.




\newpage
Example from \citet{Savva:2004}:
\begin{newicktree}
\drawtree{((My:1,first:1.5):0.5,(\sf newicktree:2,tree!:2.5):0.5):0.5;}
\end{newicktree}



\pagebreak

%---------------------------------bibliography------------------------------------

% bibtex is over-riding some formatting? Ie, the title of the Groeneveld reference is
% "... dwarf lemur species at Tsinjoarivo, eastern Madagascar." but it gets compiled as
% "dwarf lemur species at tsinjoarivo, eastern madagascar."

\bibliography{PPP_documentation}

%------------------------------------figures------------------------------------
\pagebreak

\begin{figure} %[h]
\includegraphics[width=0.9\textwidth]{figures/fig_WithOutgroup.pdf}
%\missingfigure[figwidth=6cm]{Sampling \ldots}
\caption{Effect of sampling regime on GSI values, with an outgroup. A)
Test tree 1. Groups A and B are interdigitated with each other, and
samples are added to the B3 and outgroup clades. B) Values for \gsi{A}.
C) Values for \gsi{B}. Shaded symbols indicate values that are
significant ($p<0.05$) under the permutation test.}
\label{fig:w_outgroup}
\end{figure}

\begin{figure} %[h]
\includegraphics[width=0.9\textwidth]{figures/fig_NoOutgroup.pdf}
\caption{Effect of sampling regime on GSI values, without an outgroup.
A) Test tree 2. Samples are added to the B1 and B2 clades in the same
way as for tree 1 (Fig. 1). B) Values for \gsi{A}. C) Values for \gsi{B}.
Shaded symbols indicate values that are significant ($p<0.05$)
under the permutation test.}
\label{fig:no_outgroup}
\end{figure}


\end{document}
